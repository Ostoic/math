\newcommand{\cat}[1]{\text{\textbf{#1}}}
\newcommand{\order}[1]{|#1|}
\newcommand{\inv}[1]{#1^{-1}}
\newcommand{\dual}[1]{#1^{*}}

\DeclareMathOperator{\R}{\mathbb{R}}

\DeclareMathOperator{\Mat}{Mat}
\DeclareMathOperator{\coker}{coker}
\DeclareMathOperator{\im}{im}
\DeclareMathOperator{\counit}{\varepsilon}
\DeclareMathOperator{\comult}{\Delta}
\DeclareMathOperator{\twist}{\tau}

\DeclareMathOperator{\vspan}{span}

\newcommand\restrict[1]{\raisebox{-.5ex}{$|$}_{#1}}

\let\hom\undefined
\DeclareMathOperator{\Hom}{Hom}
\DeclareMathOperator{\supp}{supp}
\DeclareMathOperator{\tr}{tr}
\DeclareMathOperator{\End}{End}
\DeclareMathOperator{\id}{id}
\DeclareMathOperator{\Vect}{\cat{Vect}}

% Switch implementation
\newcommand{\ifequals}[3]{\ifthenelse{\equal{#1}{#2}}{#3}{}}
\newcommand{\case}[2]{#1 #2} % Dummy, so \renewcommand has something to overwrite...
\newenvironment{switch}[1]{\renewcommand{\case}{\ifequals{#1}}}{}

\ExplSyntaxOn
\NewDocumentCommand{\cycle}{ O{\;} m }
 { 	
  (
  \alec_cycle:nn { #1 } { #2 }
  )
 }

\seq_new:N \l_alec_cycle_seq
\cs_new_protected:Npn \alec_cycle:nn #1 #2
 {
  \seq_set_split:Nnn \l_alec_cycle_seq { , } { #2 }
  \seq_use:Nn \l_alec_cycle_seq { #1 }
 }
\ExplSyntaxOff

\newcommand{\tensor}[1][]{
	\ifthenelse{\isempty{#1}}
	{
		\otimes
	}
%	else
	{
		\otimes_{#1}
	}
}
\newcommand{\cyclic}[1]{\langle #1 \rangle}