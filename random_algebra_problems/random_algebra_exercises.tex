\documentclass[11pt]{article}
 
\usepackage[margin=1in]{geometry} 
\usepackage{amsmath,amsthm,amssymb}
\usepackage[utf8]{inputenc}

% Algebra commands
\usepackage{xifthen}

\let\ooooldgcd\gcd

% Better gcd 
\renewcommand{\gcd}[2]{(#1, #2)}
 
 % Common number systems
\newcommand{\N}{\mathbb{N}}
\newcommand{\Z}{\mathbb{Z}}
\newcommand{\R}{\mathbb{R}}

% Group-theoretic order
\newcommand{\order}[1]{|#1|}

% Divides operator
\newcommand{\divides}{\mathbin{\text{\ divides\ }}}

% Cyclic group
\newcommand{\cyclic}[1]{\langle #1 \rangle}

% Tensor product command with optional parameter for emphasizing which ring we're working over.
\newcommand{\tensor}[1][]{
	\ifthenelse{\isempty{#1}}
	{
		\mathbin{\otimes}
	}
%	else
	{
		\mathbin{\otimes_{#1}}
	}
}

% Direct sum
\newcommand{\directsum}{\mathbin{\oplus}}


\usepackage{xifthen}

% Proof directions
% necessary is (=>)
\newcommand{\necessary}{$(\Rightarrow)$ }

%sufficient is (<=)
\newcommand{\sufficient}{$(\Leftarrow)$ }

\newcommand{\lin}[4]{ #1_{#2}, \dots #1_{#3} \in #4 }

% Basic environments
\newenvironment{definition}[2][Definition]{\begin{trivlist}
\item[\hskip \labelsep {\bfseries #1}\hskip \labelsep {\bfseries #2.}]}{\end{trivlist}}
 
\newenvironment{theorem}[2][Theorem]{\begin{trivlist}
\item[\hskip \labelsep {\bfseries #1}\hskip \labelsep {\bfseries #2.}]}{\end{trivlist}}
 
\newenvironment{lemma}[2][Lemma]{\begin{trivlist}
\item[\hskip \labelsep {\bfseries #1}\hskip \labelsep {\bfseries #2.}]}{\end{trivlist}}
 
\newenvironment{exercise}[2][Exercise]{\begin{trivlist}
\item[\hskip \labelsep {\bfseries #1}\hskip \labelsep {\bfseries #2.}]}{\end{trivlist}}
 
\newenvironment{reflection}[2][Reflection]{\begin{trivlist}
\item[\hskip \labelsep {\bfseries #1}\hskip \labelsep {\bfseries #2.}]}{\end{trivlist}}
 
\newenvironment{proposition}[2][Proposition]{\begin{trivlist}
\item[\hskip \labelsep {\bfseries #1}\hskip \labelsep {\bfseries #2.}]}{\end{trivlist}}
 
\newenvironment{corollary}[2][Corollary]{\begin{trivlist}
\item[\hskip \labelsep {\bfseries #1}\hskip \labelsep {\bfseries #2.}]}{\end{trivlist}}
 
\begin{document}
 
%\renewcommand{\qedsymbol}{\filledbox}
 
\title{Random Algebra/Number Theory Problems}
\author{Shaun Ostoic} 
 
\maketitle

\begin{exercise}{1}
Let $G$ be a finite group. Show that $a \in G$ is a generator of $G$ if and only if $a^{\frac{\order{G}}{q}} \neq 1$, for each prime factor $p$ of $\order{G}$.
\end{exercise}

\begin{proof}\necessary Since $a$ generates $G$, it is a cyclic group, hence $ G = \cyclic{a} $. A consequence of this is that $ \order{G} = \order{\cyclic{a}} = \order{a} $. Let us assume for sake of a contradiction, that $a^{\frac{\order{G}}{q}} = 1$ for some prime factor $ q $ of $ \order{G} $. Then we see that $ a^{\frac{\order{G}}{q}} = 1 \implies \order{a} \divides \frac{\order{G}}{q} \implies \order{a} = \order{G} < \frac{\order{G}}{q}$, which is a contradiction.

\sufficient Suppose instead that $ a $ is not a generator of $ G $. Then $\order{a} \neq \order{G} $, which means there must be a prime divisor $ q $ of $ \order{G} $ for which $ \order{a} \divides \frac{\order{G}}{q} $. To be more explicit, let us look at the prime factorization of $ \order{G} $ and of $ \order{a} $. Write $ \order{G} = q_{1}^{g_{1}} \dots q_{r}^{g_{r}}$ and $ \order{a} = q_{1}^{a_{1}} \dots q_{r}^{a_{r}}$, with $ a_{i} \leq g_{i} $ for all $ i $. It is entirely possible that $ a $ may hold each prime divsior of $ \order{G} $ in its prime factorization, but it does not necessarily have equal exponents in the factorization. Otherwise, $ a_{i} = g_{i} $ for all $ i $, hence $ \order{a} = \order{G} $. In this case, $ \order{a} $ and $ \order{G} $ must differ by at least one prime divisor $ q $. Then if we divide out that prime factor from $ \order{G} $, it follows that $ \order{a} \divides \frac{\order{G}}{q} $. By one of the previous theorems, this is equivalent to saying that $ a^{\frac{\order{G}}{q}} = 1 $, which contradicts the assumption we make in this direction of the proof.

\end{proof}

\begin{exercise}{2}
Show that for any prime number $ p \geq 3 $, $ \Z_{p}^{*} $ has a primitive root.
\end{exercise}

\begin{proof}
To show that $ \Z_{p}^{*} $ has a primitive root, it is sufficient to show that $ \order{a} = \phi(p) = p-1 $, for some $ a \in \Z_{p}^{*} $.
\end{proof}

\begin{exercise}{3}
Show that a finite subset $ B $ of a vector space $ V $ over a field $ F $ is a basis for $ V $ if and only if every $ v \in V $ can be written uniquely as a linear combination of vectors from $ B $. That is, where $ B = \{b_{1}, \ldots b_{n}\} $, the scalars $ \alpha_{i} \in F $ for all $ i $ are unique for which
\[ v = \alpha_{1} b_{1} + \ldots + \alpha_{n} b_{n}. \]

\end{exercise}

\begin{proof}
\necessary Suppose that $ B $ is a basis for $ V $. Then $ B $ is both a spanning set, and a linearly independent set. It is a spanning set for $ V $, which means that for every $ v \in V $, there exist scalars $ \alpha_{1}, \dots \alpha_{n} \in F $ such that 
\begin{equation} \label{eq:1}
	v = \alpha_{1} b_{1} + \dots + \alpha_{n} b_{n}.
\end{equation}
$ B $ being linearly independent means that if there exist $ \lin{\eta}{1}{n}{F} $ such that
\[ \eta_{1} b_{1} + \dots + \eta_{n} b_{n} = 0, \]
then  $ \eta_{i} = 0 $ for all $ i $.

We must now show that (\ref{eq:1}) is the only possible way to write $ v $ as a linear combination of vectors from $ B $. Suppose for sake of a contradiction that there is another such way to write $ v $, that being
\begin{equation} \label{eq:2}
	v = \beta_{1} b_{1} + \dots + \beta_{n} b_{n},	
\end{equation}
where $ \lin{\beta}{1}{n}{F} $. Using (\ref{eq:1}) and (\ref{eq:2}), we have
\[ \beta_{1} b_{1} + \dots + \beta_{n} b_{n} = \alpha_{1} b_{1} + \dots + \alpha_{n} b_{n} \]
\begin{equation} \label{eq:3}
	(\alpha_{1} - \beta_{1}) b_{1} + \dots + (\alpha_{n} - \beta_{n}) b_{n} = 0 
\end{equation}
Let us use the assumption that $ B $ is a linearly independent set. Set $ \eta_{i} = (\alpha_{i}-  \beta_{i}) $ for each $ i $. Then (\ref{eq:3}) can be written as 
\[ \eta_{1} b_{1} +  \dots + \eta_{n} b_{n} = 0, \]
which implies that $ \eta_{i} = 0 $ for all $ i $ by linear indepdendence of $ B $. That is, $ \alpha_{i} - \beta_{i} = 0 $, hence $ \alpha_{i} = \beta_{i} $ for all $ i $, which proves that these two ways of writing $ v $ are the same.

\sufficient Suppose that
\end{proof}

\begin{exercise}{4}
content...
\end{exercise}

\end{document}
	