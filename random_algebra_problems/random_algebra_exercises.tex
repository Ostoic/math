\documentclass[11pt]{article}
 
\usepackage[margin=1in]{geometry} 
\usepackage{amsmath,amsthm,amssymb}
\usepackage[utf8]{inputenc}

\let\oldgcd\gcd

\renewcommand{\gcd}[2]{(#1, #2)}
 
\newcommand{\N}{\mathbb{N}}
\newcommand{\Z}{\mathbb{Z}}
\newcommand{\R}{\mathbb{R}}
\newcommand{\dual}[1]{(#1)^*}
\newcommand{\order}[1]{|#1|}
\newcommand{\divides}{\mathbin{\text{\ divides\ }}}
\newcommand{\cyclic}[1]{\langle #1 \rangle}

\newcommand{\necessary}{$(\rightarrow)$ }
\newcommand{\sufficient}{$(\leftarrow)$ }

\newcommand{\norm}[1]{\left\lVert#1\right\rVert}

\newenvironment{definition}[2][Definition]{\begin{trivlist}
\item[\hskip \labelsep {\bfseries #1}\hskip \labelsep {\bfseries #2.}]}{\end{trivlist}}
 
\newenvironment{theorem}[2][Theorem]{\begin{trivlist}
\item[\hskip \labelsep {\bfseries #1}\hskip \labelsep {\bfseries #2.}]}{\end{trivlist}}
 
\newenvironment{lemma}[2][Lemma]{\begin{trivlist}
\item[\hskip \labelsep {\bfseries #1}\hskip \labelsep {\bfseries #2.}]}{\end{trivlist}}
 
\newenvironment{exercise}[2][Exercise]{\begin{trivlist}
\item[\hskip \labelsep {\bfseries #1}\hskip \labelsep {\bfseries #2.}]}{\end{trivlist}}
 
\newenvironment{reflection}[2][Reflection]{\begin{trivlist}
\item[\hskip \labelsep {\bfseries #1}\hskip \labelsep {\bfseries #2.}]}{\end{trivlist}}
 
\newenvironment{proposition}[2][Proposition]{\begin{trivlist}
\item[\hskip \labelsep {\bfseries #1}\hskip \labelsep {\bfseries #2.}]}{\end{trivlist}}
 
\newenvironment{corollary}[2][Corollary]{\begin{trivlist}
\item[\hskip \labelsep {\bfseries #1}\hskip \labelsep {\bfseries #2.}]}{\end{trivlist}}
 
\begin{document}
 
%\renewcommand{\qedsymbol}{\filledbox}
 
\title{Random Algebra/Number Theory Problems}
\author{Shaun Ostoic} 
 
\maketitle

\begin{exercise}{1}
Let $G$ be a finite group. Show that $a \in G$ is a generator of $G$ if and only if $a^{\frac{\order{G}}{q}} \neq 1$, for each prime factor $p$ of $\order{G}$.
\end{exercise}

\begin{proof}\necessary Since $a$ generates $G$, it is a cyclic group, hence $ G = \cyclic{a} $. A consequence of this is that $ \order{G} = \order{\cyclic{a}} = \order{a} $. Let us assume for sake of a contradiction, that $a^{\frac{\order{G}}{q}} = 1$ for some prime factor $ q $ of $ \order{G} $. Then we see that $ a^{\frac{\order{G}}{q}} = 1 \implies \order{a} \divides \frac{\order{G}}{q} \implies \order{a} = \order{G} < \frac{\order{G}}{q}$, which is a contradiction.

\sufficient Assume $ a^{\frac{\order{G}}{q}} = 1$ for every prime factor $ q $ of $ \order{G} $. Since $ \cyclic{a} $ is a subgroup of G, $ \order{\cyclic{a}} \divides \order{G}$ (by Lagrange's Theorem. Let $ m = \frac{\order{G}}{q} $. Then 
\[ \order{a^{m}} = \frac{\order{a}}{\gcd{m}{\order{a}}}. \]
We claim that $ \order{a^{m}} = q$. To see this, observe that
\[ (a^{\frac{\order{G}}{q}})^{q} = a^{\order{G}} = 1.\]
From this we see that $ \order{a^{m}} \divides q$. Since $ a^{m} \neq 1$ and $ q $ is prime, it follows that $ \order{a^{m}} = q$. Thus,
\[ \order{a^{m}} = \frac{\order{a}}{\gcd{m}{\order{a}}} = q\]
\[ \order{a} = q \gcd{m}{\order{a}}\]
\[ \implies q \divides \order{a}. \]
Since $ q $ was an arbitrary prime factor of $ \order{G} $, we see that every prime factor of $ \order{G} \divides \order{a}$.
\end{proof}

\end{document}
	